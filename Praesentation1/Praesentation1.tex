%LuaLaTeX
\documentclass[aspectratio=169]{beamer}
\usepackage[libertine,nocaption]{/Users/Shared/texdir/stdpreamble}
\usepackage{csquotes}
\usepackage{nicefrac}
\usepackage{multimedia}

\myauthor{
  Noll\and Schug\and Konrad
}
\myfooterauthor{Noll{\quad}Schug{\quad}Konrad}
\mytitle{Gestenerkennung mit Tensorflow (Home Assistant IO)}
\institute{HS KL Campus Zweibrücken}

\begin{document}

\begin{frame}
\titlepage
\end{frame}

\begin{frame}
\tableofcontents[subsectionstyle=hide]
\end{frame}

\section{Das Projekt}

\begin{frame}
Das Ziel des Projektes ist es ein Neuronales Netzwerk zu erstellen, dass mithilfe
von OpenCV die einzelnen Bilder der Kamera ausgibt und entsprechende
Gesten interpretiert. Die Gesten führen wiederum zu Aktionen von Smart Home
Geräten, die mit dem Smart Home System \enquote{Home Assistant} verbunden sind. 
\end{frame}

\section{Das Team und die Aufgaben}

\begin{frame}
  \begin{itemize}
    \item Erstellen einer Dokumentation während der einzelnen Projektphasen.
    \item Installation des Home Assistant IO auf dem Raspberry Pi
    \item Hinzufügen einer neuen Schnittstelle zur Integration eines Neuronalen
	  Netzwerkes. (Das Neuronale Netzwerk wird durch neue Gesten ergänzt und
	  anschließend auf diese trainiert.)
  \end{itemize}
  \begin{itemize}
    \item Programmierer für die Schnittstelle zu Home Assistant IO, Operationsmanager
    \item Programmierer für das Anpassen des Neuronalen Netzwerkes 
  \end{itemize}
\end{frame}

\section{Kurzdemo OpenCV}
\begin{frame}
  \vskip 0.2cm
  \movie[width=0.8\textwidth,height=0.45\textwidth,poster]{}{OpenCV.mov}
  \vfilll
\end{frame}

\section{Grobe Projektphasen}

\begin{frame}
  \begin{itemize}
    \item Theoretische Einarbeitungsphase (Dauer: ca. 2 Wochen)
    \item Zusammenfassung der Theorie + erste grobe Anpassung des Models. (Dauer: ca. 1\nicefrac{1}{2} Wochen)
    \item Hinzufügen neuer Gesten (mithilfe von angewandten Filtern) und trainieren des vorhandenen Modells, mit ersten Konditionen zur Integration von unterstützten Smart Home Geräten (Dauer: ca. 4 Wochen)
    \item Ausarbeitung der Dokumentation, Präsentation und Poster (Dauer: ca. 1\nicefrac{1}{2} Wochen)
  \end{itemize}
\end{frame}

\section{Meilensteine}

\begin{frame}
\begin{description}
  \item[16. Mai] Installation von Home Assistant IO, damit die aktive Entwicklungsphase sofort beginnen kann.
  \item[29. Mai] Durch erste Gesten sollen schon ein bis zwei Smart Home Geräte ein- und ausgeschaltet werden.
  \item[13. Juni] Ende des Projektes.
\end{description}
\end{frame}

\section{Organisation}
\picframew[Kanban Board]{Board}

\end{document}
